\documentclass[a4paper,12pt]{article}
\usepackage[utf8]{inputenc}
\usepackage[brazilian]{babel}
\usepackage[T1]{fontenc}
\usepackage[a4paper,left=30mm,right=30mm,top=15mm,bottom=15mm]{geometry}
\usepackage{graphicx,url}
\usepackage{color,comment, pifont}
\usepackage{amssymb}

\begin{comment}
\usepackage[T1]{fontenc}
\usepackage[utf8]{inputenc}
\usepackage{lmodern}
\usepackage[francais]{babel}
\end{comment}

\title{Lógica Matemática -- Trabalho Final -- 2017-2}
\author{Claudio Cesar de Sá e Rogério Eduardo da Silva}
\date{\today}

\graphicspath{{/figures/}}   
\DeclareGraphicsExtensions{{.jpg},{.png}}


\begin{document}
\maketitle

\begin{flushleft}


\vspace{0.5cm}
\ding{224}  {\bf \textcolor{red}{
Antes de tudo leia com \textbf{muita atenção}.} Em geral, há muitos equívocos
que os alunos cometem por não lerem corretamente!}


\vspace{0.5cm}
\ding{224} {\bf \textcolor{blue}{Então: leiam atentamente as
instruções que se seguem!}}


\vspace{0.5cm}
\ding{224}  {\bf \textcolor{red}{Os enunciados dos problemas encontram-se no site oficial dos problemas escolhidos.}}


\vspace{0.5cm}
\ding{224}  {\bf \textcolor{red}{Este arquivo vai estar sempre atualizado em: }}\\
{\bf \textcolor{red}{\url{https://github.com/claudiosa/CCS/tree/master/picat/TRABALHOS_FINAIS/}}}

\vspace{0.5cm}
\ding{224} Tarefa: Implementar todos  
 problemas propostos abaixo e os exercícios, em detalhes, que se seguem. 
 %%Peso de cada tarefa: $\frac{1}{3}$

\vspace{0.5cm}
\ding{224} Entrega pelo site: \textcolor{red}{\url{https://www.cloudwok.com/u/O1D1}}


\vspace{0.5cm}
\ding{224}  Este site  \textcolor{red}{\url{https://www.cloudwok.com/u/O1D1}} é NOVO, siga as instruções para \emph{upload}. Proceda até
a mensagem \textbf{ \emph{Your message has been sent!}}
%\vspace{0.5cm}
%\ding{224} A senha de entrega  é: \texttt{lma} (as siglas da %disciplina em letras minúsculas)

\vspace{0.5cm}
\ding{224} Entrega dos trabalhos: \textcolor{red}{xx/junho} (para o 1o. Semestre)\\
\textcolor{red}{27/novembro} (para o 2o. Semestre). Em geral, pode-se
ocorrer uma flexibilização aqui.\\

%\vspace{0.5cm}
%\ding{224} Todas estas datas foram escritas no primeiro dia letivo de aula
%do semestre corrente junto com datas de provas.


\vspace{0.5cm}
\ding{224} Implementação em SWI-Prolog, Eclipse (www.eclipseclp.org) ou Picat\\

\vspace{0.5cm}
\ding{224} \textcolor{red}{\textbf{Quanto aos nomes dos arquivos a serem enviados}}:
\begin{itemize}
  \item \textbf{Não envie os arquivos compactados} (serão automaticamente excluídos)
  \item Envie os arquivos  via o site: \textcolor{red}{\url{https://www.cloudwok.com/u/O1D1}}

  \item Não use email para enviar aos professores
  
  \item O nome do arquivo deste deve conter: seu nome,
  sua turma, e o problema resolvido, extensão pode ser txt, pl, ecl, pi etc.
  \item Não coloque espaços em brancos nos nomes do problemas. Use o '\_'  (\textit{underscore}) para ligar nomes
  \item Exemplo de nome de uma arquivo: \\ \textbf{joao\_silva\_e\_pedro\_souza\_TB\_problema\_das\_estrelas.txt}
  \item Dentro dos códigos coloque o seu nome também.
\end{itemize}


\vspace{0.5cm}
\ding{224} Além dos códigos, sob forma de comentários as 
entradas e saídas com os testes de seus programas. Estas
entradas e saídas devem vir COMENTADAS no código fonte.

\vspace{0.5cm}
\ding{224} Os testes exaustivos no próprio código fonte vão demonstrar que seu programa está fazendo o que se solicita.

\vspace{0.5cm}
\ding{224} Inclua a saídas do programa e seu tempo de execução (\textbf{\textcolor{green}{isto vai assegurar que não existam cópias de código}}). Há um exemplo de como se calcula tempo de execução, ver código: \textbf{hexagono\_19.ecl}

\vspace{0.5cm}
\ding{224} Alguns fontes e materiais de apoio (incluindo este enunciado) estão em: \textbf{\url{https://github.com/CCS/picat}} 

\vspace{0.5cm}
\ding{224} \textbf{\textcolor{magenta}{Não se impressione pela classificação da dificuldade do problema no site. O que é difícil para o homem, pode ser fácil para máquina!}}

\end{flushleft}


\begin{enumerate}
\setlength\itemsep{0.1cm}
\item Dicas de como se resolve manualmente:\\
\url{http://www.valdiraguilera.net/problema-de-logica-esquema.html}

\item Há exemplos detalhados para estudo em:
\begin{itemize}
  \item  \url{https://github.com/CCS/prolog} 
  \item  \url{https://github.com/CCS/picat}
\end{itemize}



%\item Use a lista da disciplina para as dúvidas ou procure 
%  os professores \textbf{pessoalmente}

\item Para que o \textit{código de honra} (evitar cópias de trabalhos) seja mantido, troquem os nomes dos personagens das estórias abaixo, por seus nomes e/ou de suas família/amigos etc. 

\end{enumerate}
%%%%%%%%%%%%%%%%%%%%%%%%%%%%%%%%%%%%%%%%%%%%%%%%%%%%%%%%%%%%

\newpage

\begin{center}
\fbox{\fbox{{\Huge \textbf{\hskip 1cm AVISO\hskip 1cm}}}}

\vskip 2cm
{\Large
Para todos quando formos ao laboratório: \textbf{\textcolor{magenta}{nem pensem em atacar estes problemas de imediato}}. Poderá ser frustrante para alguns. Voces deverão começar com os exercícios de sala de aula e os do site. \textbf{Um passo de cada vez !}
}
\vskip 2cm
\end{center}

Algumas fontes alternativas de aprendizado s\~ao:

\begin{enumerate}

%\item  \url{https://www.dropbox.com/home/cursos/lma/exercicios_prolog}

\item Alguns outros Prologs:  \url{http://www.thefreecountry.com/compilers/prolog.shtml}

\item Prolog on-line: \url{http://www.tutorialspoint.com/execute_prolog_online.php}. Simplesmente: \textbf{\textcolor{magenta}{Fantástico!}}

\item PICAT on-line: \url{http://picat.retina.ufsc.br/picat.html}. Simplesmente: \textbf{\textcolor{magenta}{Fantástico!}}



 \item No seu telefone (\emph{smartphone}) instale: Jekejeke Prolog (nenhuma semelhança com o time local), tanto faz o Runtime ou o Development (este vem com 
 \emph{debugger}, ótimo para aprender de verdade)
 
 \item Ver os vídeos no Youtube no canal do Prof. \texttt{Claudio Cesar de Sa} referente
 a resolução de problemas no Racha-Cuca

\end{enumerate}





\newpage
\tableofcontents


%%%%%%%%%%%%%%%%%%%%%%%%%%%%%%%%%%%%%%%%%%%%%%%%%%%%%%%%%%%%%%%%%%%%%%%%%%%%%
\newpage
\section{Residencial Cientistas Famosos}


Descubra as características dos cinco prédios e dos seus 
respectivos arquitetos do residencial Cientistas Famosos.

Como as férias estão se aproximando ... nada como um passeio
pelo conjunto de residencia de Cientistas Famosos, 
 para voce 
relembrar as aulas de LMA, eis o problema proposto:\\
 Fonte do problema proposto:\\
 \url{https://rachacuca.com.br/logica/problemas/residencial-cientistas-famosos/}
 (tem a montagem da tabela para irem entendendo e depurando o problema).\\


\vspace{1.5cm}
\ding{224} Sua tarefa é associar todas essas informações a partir dessas dicas dadas e deduzir o que problema solicita. Acompanhe o andamento de sua solução pela fornecida no site.
%%%%%%%%%%%%%%%%%%%%%%%%%%%%%%%%%%%%%%%%%%%%%%%%%%%%%%%%%%%%%%%%%%%%%%%%%%%%%
\newpage
\section{Feira de Antiguidades}

Cinco colecionadores estão visitando uma feira de antiguidades em busca de novos itens para colecionar. 
Descubra as características deles usando a lógica
Ainda relacionado as férias, eis o problema proposto a voce e sua turma:\\
 Fonte do problema proposto:\\
 \url{https://rachacuca.com.br/logica/problemas/feira-de-antiguidades/}
 (tem a montagem da tabela para irem entendendo e depurando o problema).\\


\vspace{1.5cm}
\ding{224} Sua tarefa é associar todas essas informações a partir dessas dicas dadas e deduzir o que problema solicita. Acompanhe o andamento de sua solução pela fornecida no site.
%%%%%%%%%%%%%%%%%%%%%%%%%%%%%%%%%%%%%%%%%%%%%%%%%%%%%%%%%%%%%%%%%%%%%%%%%%%%%
\newpage
\section{Recursividade -- Exercicios}


Implemente em um único arquivo, os seguintes exercícios
de recursividade:

\begin{enumerate}
\item Encontre o n-ésimo  termo da seguinte  sequência recursiva:  $1, 4, 8, 13, 19, 26,...$. 

\item Encontre o n-ésimo  termo e o seguinte (ou seja, dois termos), para  sequência recursiva dada por: $2, 3, 3, 5, 10, 13, 39,...$. Dica:
\begin{verbatim}
2 + 1 = 3
3 x 1 = 3
3 + 2 = 5
5 x 2 = 10
10 + 3 = 13
13 x 3 = 3
\end{verbatim}

\item  Giuseppe Peano foi uma matemático que observou que os números
naturais podem ser obtidos por  somas recursivas, dadas por:
$x + 0 = x$ e $x + soma(y) = soma(x + y)$. Traduza isto para 
LPO e na sequência escreva em Picat (ou Prolog) que um dado número $N>0$ este pode ser provado como sendo um número natural. Referência: postulados aritméticos de Peano

\item Aproveitando o fatorial recursivo feito em sala, implemente a seguinte soma recursiva para $n$ termos, dado um expoente  $x$.
  $$e^x = \sum^\infty_{n=0} {x^n\over n!} =1 + x + {x^2 \over 2!} + {x^3 \over 3!} + {x^4 \over 4!}+\cdots.$$
  em termos práticos, se x = 1, tem-se:
$e^1 \approx 2.718281828459045...$ se $n$ for muito grande (o que pode demorar muito no seu computador)! Outro exemplo:
$e^5 \approx 148.4131591025766...$. Então teste para $n$ pequeno como $n < 10$.

\vspace{0.7cm}
Para referência deste problema: veja \url{https://en.wikipedia.org/wiki/E_(mathematical_constant)}. Constante neperiana, fórmula deduzida por Euler, assim a constante logaritmica dos números naturais, $e$, ficou conhecida como número de Euler!

\end{enumerate}
%%%%%%%%%%%%%%%%%%%%%%%%%%%%%%%%%%%%%%%%%%%%%%%%%%%%%%%%%%%%%%%%%%%%%%%%%%%%%



\end{document}
