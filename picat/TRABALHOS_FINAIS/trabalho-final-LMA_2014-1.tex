\documentclass[a4paper,12pt]{article}
\usepackage[utf8]{inputenc}
\usepackage[brazilian]{babel}
\usepackage[T1]{fontenc}
\usepackage[a4paper,left=30mm,right=30mm,top=15mm,bottom=15mm]{geometry}
\usepackage{graphicx,url}
\usepackage{color,comment, pifont}
\usepackage{amssymb}

\begin{comment}
\usepackage[T1]{fontenc}
\usepackage[utf8]{inputenc}
\usepackage{lmodern}
\usepackage[francais]{babel}
\end{comment}

\title{Lógica Matemática -- Trabalho Final -- 2014-2}
\author{Rogério Eduardo da Silva, Claudio Cesar de Sá\\}
\date{\today}

\graphicspath{{/figures/}}   
\DeclareGraphicsExtensions{{.jpg},{.png}}


\begin{document}
\maketitle

\begin{flushleft}

\ding{224} Tarefa: Implementem os  03 (três)  problemas dos problemas propostos abaixo.\\
Entrega: definir data e forma com Prof. Rogério \\ 

\ding{224} Implementação em SWI-Prolog ou Eclipse (www.eclipseclp.org) (ver a apostila do curso)\\

\ding{224} Além dos códigos, sob forma de cometários as entradas e saídas com os testes de seus programas\\

\ding{224} Os testes exaustivos no próprio código fonte vão demonstrar que seu programa está fazendo o que se solicita.


\ding{224} Alguns fontes e materiais de apoio estão em:
 \url{https://www.dropbox.com/home/cursos/lma/exercicios_prolog} ou \\
  \url{http://www2.joinville.udesc.br/~coca/index.php/Main/LogicaMatematica}

\end{flushleft}

\begin{enumerate}
\setlength\itemsep{0.5cm}
\item Dicas de como se resolve manualmente:\\
\url{http://www.valdiraguilera.net/problema-de-logica-esquema.html}

\item Há exemplos bem detalhados do semestre passado\\

\item Use a lista da disciplina para as dúvidas ou procure
os professores \textbf{pessoalmente}\\

\item Para que o \textit{código de honra} (evitar cópias de trabalhos) seja mantido,
troquem os nomes dos personagens das estórias abaixo, por seus nomes e/ou de suas família/amigos etc. 

\end{enumerate}

\newpage
\tableofcontents



\newpage
\section{Presentes de Natal}

 Fonte do problema proposto:\\
 \url{http://rachacuca.com.br/logica/problemas/presentes-de-natal/}
 (tem a montagem da tabela para irem entendendo e depurando o problema).\\

{\em 
Descubra o que cada um dos cinco meninos quer ganhar de natal.
\begin{enumerate}
 %%  \item    \textbf{\textcolor{red}{Falta o enunciado dos FATOS}}
\item	O garoto que quer ser Bombeiro mora em alguma das casas das pontas.
\item	Quem gosta de Maracujá quer ser Médico.
\item	Na terceira casa está o garoto que quer ser Professor.
\item	Em uma das pontas mora o menino que quer ser Policial.
\item	Na quinta casa está o garoto que deseja ser Bombeiro quando crescer.
\item	O menino que gosta de suco de Laranja está em uma das pontas
\item	O garoto que gosta de limonada está exatamente à esquerda do que gosta de suco de Morango.
\item	Quem quer ser Médico está exatamente à direita de quem gosta de suco de Abacaxi.
\item	Cristian gosta de suco de Limão.
\item	O menino de 8 anos mora ao lado do que quer um Computador de presente.
\item	O garoto que quer um Skate está exatamente à esquerda da casa do Alex.
\item	Na casa de cor Branca mora o menino que deseja um Vídeo Game de natal.
\item	Na primeira casa mora o garoto que quer uma Bola de presente de natal.
\item	Quem quer uma Bicicleta mora ao lado de quem deseja um Vídeo Game de presente.
\item	O menino mais novo mora na quinta casa.
\item	O garoto de 10 anos está em algum lugar entre o de 7 e o de 9 anos, nessa ordem.
\item	O garoto de 9 anos quer ser Professor quando crescer.
\item	Pedro mora na terceira casa.
\item	Eduardo mora exatamente à direita de Pedro.
\item	Eduardo mora na casa de cor Verde.
\item	A casa de cor Branca está exatamente à esquerda da casa do menino que gosta de suco de Maracujá.
\item	A casa de cor Azul está exatamente à esquerda da casa do garoto que quer um Vídeo Game.
\item	A primeira casa é Amarela.
   
\end{enumerate}

Siga as dicas e encontre o nome, a idade e o suco favorito. Encontre também qual o presente que cada um deseja de natal e qual a profissão que eles desejam ter quando crescer.


\newpage
\section{Passeio no Zoológico}

 Fonte do problema proposto:\\
 \url{http://rachacuca.com.br/logica/problemas/passeio-no-zoologico/}
 (tem a montagem da tabela para irem entendendo e depurando o problema).\\

{\em 
Cinco amigas estão esperando na fila para entrar no zoológico. Descubra as preferências de cada uma delas.
\begin{enumerate}
 %%  \item    \textbf{\textcolor{red}{Falta o enunciado dos FATOS}}
\item   A penúltima garota da fila gosta de Biologia.
\item	A menina de mochila Amarela está em algum lugar à esquerda da garota que gosta de Português.
\item	A menina que gosta de História está em algum lugar entre a Joana e a menina que gosta de Geografia, nesta ordem.
\item	A Pati está exatamente à esquerda da garota que quer ver a Girafa.
\item	A garota que que ver a Arara está ao lado da que quer ver o Leão.
\item	A menina da mochila Vermelha está em algum lugar à esquerda da que quer ver o Leão.
\item	A garota que quer ver o Macaco usa uma mochila Branca.
\item	A menina que trouxe uma Maçã está na quarta posição da fila.
\item	A garota que gosta de Geografia vai comer Salgadinho.
\item	A menina que trouxe Chocolate está na segunda posição.
\item	A garota que vai comer Sanduíche gosta de Português.
\item	A menina que gosta de suco de Morango está em uma das pontas da fila.
\item	A menina que gosta de suco de Limão está entre a garota que quer ver o Elefante e a que gosta de suco de Maracujá, nesta ordem.
\item	A garota da mochila Amarela gosta de tomar limonada.
\item	A menina que gosta de História gosta de suco de Laranja.
\item	A garota que prefere suco de Abacaxi está em uma das pontas da fila.
\item	A menina que gosta de suco de Morango está exatamente à direita de quem gosta de suco de Maracujá.
\item	A Renata está entre a menina que vai comer Banana e a Ana, nesta ordem.
\item	A Jéssica ocupa a segunda posição na fila.
\item	O suco preferido da Pati é o de Limão.
\item	A garota que gosta de Geografia está ao lado da de mochila Azul.
\item	A menina da mochila Vermelha está exatamente à direita da de mochila Verde.
   
\end{enumerate}

Deduza o nome, suco, lanche, animal e matéria preferida de cada uma das cinco amigas. Elas estão aguardando numa fila para entrar no zoológico.



\newpage
\section{Amigas no Cinema}

 Fonte do problema proposto:\\
 \url{http://rachacuca.com.br/logica/problemas/amigas-no-cinema/}
 (tem a montagem da tabela para irem entendendo e depurando o problema).\\

{\em 
Descubra o nome, idade, cor favorita, estilo de filme e o nome do namorado de cada uma das cinco amigas.
\begin{enumerate}
 %%  \item    \textbf{\textcolor{red}{Falta o enunciado dos FATOS}}
\item   A mulher com 26 anos está sentada ao lado da que curte filmes de Ficção.
\item	A Lilian está sentada ao lado da quem gosta de filmes de Ação.
\item	A que namora o Adriano não está sentada ao lado da mulher que gosta de filmes de Ficção.
\item	Quem gosta de Comédia está sentada exatamente à esquerda da que tem 26 anos.
\item	Quem gosta de Drama está sentada em uma das pontas.
\item	A moça com 23 anos está sentada exatamente à direita da que curte o estilo Drama.
\item	A Lilian está sentada exatamente à esquerda da sua amiga que gosta da cor Branca.
\item	Quem gosta de verde está sentada em uma das pontas.
\item	A namorada do Willian está sentada imediatamente à esquerda da sua amiga com 27 anos.
\item	A moça que tem 22 anos está sentada exatamente à direita da namorada do Marcelo.
\item	Quem tem 22 anos está sentada em algum lugar à direita da namorada do Willian.
\item	A Lilian está sentada à esquerda de quem namora o Flávio.
\item	A namorada do Adriano tem 25 anos de idade.
\item	A Fernanda adora a cor amarela.
\item	A Rosana está sentada na cadeira mais a direita.
\item	A Luciana está sentada ao lado da sua amiga de 27 anos.
\item	As garotas que curtem amarelo e vermelho estão sentadas na quarta e quinta posições, respectivamente.
\item	Quem gosta de ver filmes de Romance não está sentada ao lado de quem curte a cor Azul.
   
\end{enumerate}

Nesse problema temos cinco amigas que foram assistir um filminho no cinema juntas. Elas estão sentadas uma ao lado da outra e cada uma tem um gosto diferente pra filmes e cores. Além disso, elas tem idades diferentes e namoram rapazes com diferentes nomes.

Sua tarefa é associar todas essas informações a partir das dicas dadas.


\end{document}
