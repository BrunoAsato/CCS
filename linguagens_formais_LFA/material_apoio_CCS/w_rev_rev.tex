\documentclass[a4paper,12pt]{article}
\usepackage[utf8]{inputenc}


\title{LFA}
\author{Prova por Indução Matemática: Reverso do Reverso}
\date{1 de Março de 2018}
\begin{document}

\maketitle

\begin{enumerate}
	\item Sabendo-se que $u^rv^r=(vu)^r$. Prove $(w^r)^r=w$.\\\\
	
	
	BASE:
	\begin{itemize}
		
	
		
		\item $k=|w|$
		\item $k=0 \rightarrow \Lambda^r = \Lambda \rightarrow (\Lambda^r)^r = \Lambda$
		\item $k=1 \rightarrow a^r = a \rightarrow (a^r)^r = a$\\
		.\\
		.\\
		.\\
		
		
			
	\end{itemize}
	
	HIPÓTESE INDUTIVA:
	\begin{itemize}
		\item $k \rightarrow (w^r)^r=w $\\
		

		\end{itemize}
			\textbf{$\rightarrow$ Precisamos provar para (k+1) que $((aw)^r)^r = aw $}  \\\\
	
	PROVA:
	\begin{itemize}

		
		 \item$((aw)^r)^r=(w^ra^r)^r$ \space\space\space\space\space\space\space\space\space\space\space\space\space\space\space\space 
		 (1) Teorema $(uv)^r=v^ru^r$
		 
		 \item$(w^ra^r)^r=(xy)^r$ \space\space\space\space\space\space\space\space\space\space\space\space\space\space\space\space\space\space\space\space 
		 (2) $x=w^r$ e $y=a^r$
		
		\item $(xy)^r = y^rx^r$\space\space\space\space\space\space\space\space\space\space\space\space\space\space\space\space\space\space\space\space\space\space\space\space\space 
		(3) Teorema do passo (1)
		 
		 \item $ y^rx^r=(a^r)^r(w^r)^r$ \space\space\space\space\space\space\space\space\space\space\space\space\space\space\space\space\space\space 
		 (4) Hipótese Indutiva
		 
		 \item $a(w^r)^r = aw$ \space\space\space\space\space\space\space\space\space\space\space\space\space\space\space\space\space\space\space\space\space\space\space\space 
		 (5) Hipótese Indutiva \\
		
		 \item $((aw)^r)^r = aw$ \space\space\space\space\space\space\space\space\space\space\space\space\space\space\space\space\space\space\space\space\space\space (6) C.Q.D.
		 
		 
		 

		 
		 
			
\end{itemize}	
		

	
	
\end{enumerate}



\medskip


\end{document}
