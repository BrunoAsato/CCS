% Modelo LaTeX, de 2012, para confecção de TCC do BCC/DCC/CCT/UDESC 
% (sem figuras)
%   Agradecimentos aos autores André Körbes (2007) e Camilla Heleno (2012)
%
% Observação importante:
%   Não é fiel às normas de 2013, mas é uma boa aproximação! Contribuições para
%   seu aperfeiçoamento são muito bem-vindas.
%   Favor enviá-las a:
%     Alexandre Gonçalves Silva <alexandre@joinville.udesc.br>
%
% O PDF final gerado por este modelo pode ser obtido na base Pergamum da UDESC:
%   http://www.pergamumweb.udesc.br/dados-bu/000000/000000000017/0000175B.pdf 



\NeedsTeXFormat{LaTeX2e}
%-----------------------------------------------------------
\documentclass[a4paper,12pt]{monografia}
\usepackage{amsmath,amsthm,amsfonts,amssymb}
\usepackage[mathcal]{eucal}
\usepackage{latexsym}
\usepackage{isoaccent}
\usepackage[portuges,brazil,brazilian]{babel}
\usepackage[utf8]{inputenc}
\usepackage{bm}
\usepackage[alf]{abntcite}
\usepackage{graphicx}
\usepackage{epstopdf}
\usepackage{subfigure}
\usepackage{textcase}
\usepackage[portuguese,noend,ruled]{algorithm2e}
\usepackage{listings}
\usepackage{multirow}

%%%%%%%% LISTINGS

\lstset{
	breaklines=true,
	basicstyle=\footnotesize\ttfamily,
	frame=single,
	keywordstyle=\bf,
	%language=C++,
	numbers=left,
	numbersep=5pt,
	tabsize=4
}

%\usepackage{color}
%\definecolor{colBack}{rgb}{1,1,.98}
%\definecolor{colKeys}{rgb}{0,0,0}
%\definecolor{colIdentifier}{rgb}{0,0,0.9}
%\definecolor{colComments}{rgb}{.4,.4,.4}
%\definecolor{colString}{rgb}{0,0,0.6}
%\lstset{language={},basicstyle=\ttfamily\small,tabsize=8,%
%frame=single,showtabs=false,showspaces=false,numbers=left,firstnumber=auto,%
%numberstyle=\tiny,linewidth=0.98\linewidth,xleftmargin=21pt,%
%float=tbph,extendedchars,breaklines,showstringspaces=false,%
%backgroundcolor=\color{colBack},columns=flexible,captionpos=b,%
%aboveskip=\bigskipamount}

%\lstset{basicstyle=\tiny,language=C++}

%-----------------------------------------------------------
%-----------------------------------------------------------
\begin{document}

%
%----------------- Título e Dados do Autor -----------------
\titulo{Construção de uma Biblioteca de Grafos para a Linguagem MiniZinc}
\autor{Marcos Creuz Filho}
\nome{Marcos}
\ultimonome{Creuz}

%
%---------- Informe o Curso e Grau -----
\bacharelado \curso{Ciência da Computação} \mes{Dezembro} \ano{2015} 
\data{\today} % data da aprovação
\cidade{Joinville}
%
%----------Informações sobre a Institução -----------------
\instituicao{Universidade do Estado de Santa Catarina}
\sigla{UDESC} \unidadeacademica{Centro de Ciências Tecnológicas}
%
%-------- Informações obtidas na Biblioteca ----------------
%\CDD{536.7}
%\CDU{536.21}
%\areas{1.Análise Matemática - Teoria das
%        Funçõees 2. Funções Analíticas - Fórmula de Taylor.}
%\npaginas{45}  % total de páginas do trabalho
%------Nomes do Orientador, 1o. Examinador e 2o. Examinador-
\orientador{Claudio Cesar de Sá}

\examinadorum{Carlos Norberto Vetorazzi Junior}

\examinadordois{Omir Alves Correa Junior}

\examinadortres{Diego Buchinger}

%--------- Títulos do Orientador 1o. e 2o. Examinadores ----
\ttorientador{Doutor}

\ttexaminadorum{Mestre}

\ttexaminadordois{Doutor}

\ttexaminadortres{Mestre}

\maketitle
%---------------------- AGRADECIMENTO --------------
%\agradecimento{Dedicatória}
%\newpage
%---------------------- EPÍGRAFE --------------
%\begin{epigrafe}

%\end{epigrafe}
%--------Digite aqui o seu resumo em Português--------------
\resumo{Resumo}
\input{chapter/0/resumo}
%-----------Digite aqui o seu resumo em Inglês--------------
\resumo{Abstract}
\input{chapter/0/abstract}
%-----------------------------------------------------------


% lista de abreviações
%\listofabbreviations{Lista de Abreviaturas}
%\begin{table}[ht]
%\begin{tabular}{ll}
%	{\bf XML } & {\it eXtensible Markup Language} \\
%\end{tabular}
%\end{table}

%\listoftables

%----Sumário, lista de figura e de tabela ------------
\tableofcontents 
\thispagestyle{empty} 
\thispagestyle{empty} \listoffigures
%\listoftables \thispagestyle{empty}

%---------------------
%--------------Início do Conteúdo---------------------------
\pagestyle{ruledheader}

\input{chapter/1/mainfile}
\input{chapter/2/mainfile}
\input{chapter/3/mainfile}
\input{chapter/4/mainfile}
\input{chapter/5/mainfile}
% \input{chapter/6/mainfile}

\renewcommand{\bibname}{Referências}
\bibliographystyle{abnt-alf}
\bibliography{tcc}

%\include{apendice}

\end{document}
