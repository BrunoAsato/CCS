
\chapter{Complexidade de Problemas}
\label{cp:complexidade}


Neste capítulo são apresentados os fundamentos sobre
complexidade de problemas.  Avaliar o que são problemas
exponenciais e o quanto são complexos.



\section{Mapeando o Territ\'orio}


Os problemas combinatoriais são ubíquos no cotidiano de pessoas, indústria, etc.
Basicamente utiliza-se alguma estimativa entre combinar objetos,
pessoas versus recursos para se atingir algum objetivo.
Em termos matemáticos, tem-se variáveis tais como $x_1$, $x_2$, $x_3$, etc, a terem
seus valores instanciados sobre algum domínio ou terem alguma relação
com outras variáveis, segundo algum critério ou requisito.

Exemplificando, voce quer organizar um calendário de jogos para os times da
sétima divisão de sua cidade. Assim, os $n$ times devem fazer um rodízio
nos $2$ campos disponíveis, tal que todo sábado  tenham $2$ jogos
por campo. Logo, serão $8$ times jogando num final de semana. Basicamente
voce deve distribuir estes encontros entre dois times respeitando critérios
como:
\begin{enumerate}
\item Local do jogo: todos os times devem jogar igualmente nos dois campos.
Afinal, um deles tem grama e a bola \textit{corre redonda};
\item Espaçamento entre os encontros: os times devem jogar até o final
da temporada;
\item Devem alternar o horário de jogo: afinal, se seu time tiver o primeiro
jogo da tarde, a feijoada está comprometida.
\end{enumerate}

Enfim, este simples ensaio mostra o cotidiano de situações combinatoriais
em que tudo se apresenta. Estes problemas vem sendo atacados há séculos tendo 
Euler como um dos matemáticos precursores, ao analisar o problema das 7 pontes 
das  Königsberg. A representação por meio de grafos foi uma estratégia
de analisar e solucionar o problema. O problema do caminho euleriano é
 base de vários estudos, pois sua  tratabilidade o coloca na classe
 de problemas NP \cite{sipser12}.

Neste encaminhamento, várias áreas de pesquisa  estão focadas como: Pesquisa
Operacional, Computação Evolutiva, Inteligência Artificial (IA), Programação por
Restrições (PR), etc. O contexto desta pesquisa encontra-se dentro destas duas 
últimas áreas, mais especificamente a \PR .


\subsection{Problemas em Inteligência Artificial}

Diversas áreas estão associadas à Inteligência Artificial como tradução de línguas, interpretação de regras, robótica em geral,  manipulação de dados em imagens,
 planejamento, reconhecimento  e outras atividades. Uma das caraterísticas destes problemas 
é a sua complexidade algoritmica e computacional, que em geral crescem exponencialmente.
Dado este fato, há uma dificuldade inerente na tarefa do desenvolvimento de algoritmos
 e solucoes aceitáveis. Estes aspectos se relacionam a área de tratabilidade de problemas,
 os quais delineam as fronteiras dos limites dos problemas em aceitarem
 soluções em um tempo razoáveis de processamento.


Para encontrar uma solução  aos problemas abordados pela IA é necessário que o processo de análise seja detalhado e modelado de 
acordo com técnicas específicas. Neste sentido duas áreas da IA são destacadas \cite{RusNorv2010}: \textit{representação do 
conhecimento}  e aplicação de \textit{métodos de busca}. Estas duas áreas apresentam uma interseção a \PR  pois os problemas quando 
atacados com PR necessitam apresentar um \textit{modelo} a ser computado e esquemas
de buscas sobre os espaço de estados que o mesmo exibe. Um dos  resultados
de uma solução com PR é a construção deste modelo, sob o qual as restrições serão
postadas, e avaliadas segundo um mecanismo \textit{completo} de busca.
 Em PR o tema da modelagem 
é investigado em \cite{smith_2006} e buscas em \cite{beek_2006,hoos_2006}.

%%% There are three main algorithmic techniques for solving constraint satisfaction problems:
%%% backtracking search, local search, and dynamic programming. 

%%% MR Jones, Patinoeire, clermont, 17:21, 24/out

\subsection{Problemas de Satisfação de Restrições}
\label{sec:ch_cp}

Um problema combinatorial cl\'assico é apresentado por um conjunto de vari\'aveis de um sistema,  as quais  ser\~ao instanciadas por objetos de domínios, segundo um conjunto de relaç\~oes, as quais representam o relacionamento entre os objetos. A tarefa combinatorial
é dada pela aç\~ao de instanciar estes objetos as variáveis, de tal modo que todas as 
relaç\~oes sejam satisfeitas.

A esta classe de problemas combinatoriais é conhecida como \textit{Problemas de Satisfação de Restrições} (PSRs). A resoluç\~ao dos PSR's constituem em encontrar valores as variáveis
respeitando ou satisfazendo suas restrições. Para esta classe de problemas lança-se mão do uso da \textit{Programação por Restrições} (PPR ou PR), ou seja, uma técnica que utiliza uma teoria e  ferramentas pr\'oprias de  programação. A PPR é uma forma de aplicar os conceitos de variáveis, domínios e restrições, via esta teoria específica de programação. 

Um PSR é tipicamente um problema \textit{NP-Completo} \cite{rossi_2006}. O desafio de todo o processamento por restrições está em gerar algoritmos que resolvam esta classe de problemas em um tempo
computacional aceitável. Invariavelmente, alguns destes problemas NP, podem apresentar uma
complexidade espacial consider\'avel, assim passam para classe P-SPACE \cite{sipser12}.
Dado este aspecto combinatorial e de complexidade NP, esta passa ter interesse
por outras àreas da pesquisa que lidam buscas \textit{heurísticas}, tais como a Computação Evolutiva (CE),
 ou ainda buscas \textit{completas} com a IA clássica e a Pesquisa Operacional (PO), etc. Esta situaçao é ilustrada pela figura \ref{fig:eureka}.


\begin{figure}[!ht]
\begin{center}
%%%  \includegraphics[scale=0.5]{figuras/psr_01.eps}
  \caption{Família dos problemas do tipo satisfação de restrições}
\label{fig:eureka}
\end{center}
\end{figure}


A \PR por sua vez, a exemplo da IA, CE, PO, etc, apresenta várias outras
subdivisões e sub-áreas de interesse.  Uma delas ela é a \textit{Programação em Lógica com Restrições} (PLR) um dos segmentos a serem atacados nesta pesquisa. A PLR tem na lógica de primeira-ordem 
o seu modelo computacional, o qual é calculado a partir de buscas exaustivas sobre
os seus \textit{modelos consistentes} (modelos de Herbrand).

Uma das   partes mais instigantes sobre os PSRs é que os mesmo são onipresentes
em problemas do mundo real. Alguns destes  problemas são discutidos em \cite{rossi_2006}.
Destancam-se os problemas de escalonamento, planejamento, roteamento, contenção,
alocação, etc.

As restrições podem ser consideradas como informações e dados há um programa por restrições. 
Estas  visam limitar o \textit{espaço de busca} e descrevem propriedades de 
variáveis/objetos e o relacionamento entre eles. As restrições são formalizadas como uma relação entre os objetos e esses são modelados como variáveis \cite{fruewirth2003}.

A relação existente entre os problemas de satisfação de restrições e a programação por restrições é expressa pela figura \ref{fig:eureka}. Portanto, pode-se considerar que a PR está contida em PSR. Ou seja, a PR é um método que pode ser aplicado para encontrar a solução de problemas do tipo PSR.

Na PR, se a abordagem for feita via programação em lógica, então tem-se a PLR. A PLR é atrativa sob os seguintes requisitos metodológicos:

\begin{itemize}
\item Adequação a representação do conhecimento, caso este seja construído em lógica formal;
\item Rápida prototipação e consequentemente baixo custo de desenvolvimento;
\item Visão declarativa de suas restrições, possibilitando uma facilidade quanto aos testes e depuração;
\item Flexibilidade na codificação dos algoritmos por abstrair características de programação em lógica.
\end{itemize}

Contudo, a Programação em Lógica com Restrições é resultado da utilização do paradigma da programação em lógica somado à programação por restrições. A PLR está contida no subconjunto de técnicas que fazem uso da PL para resolver problemas do tipo PSR. As vantagens de se utilizar a programação em lógica por restrições (PLR) estão em: modelar problemas de forma declarativa com uma sólida base matemática, propagação dos efeitos das decisões utilizando algoritmos eficientes e busca por soluções ótimas \cite{fruewirth2003}. Desde o início dos anos 90, a programação baseada em restrições tem tido sucesso comercial e industrial \cite{RusNorv2010}. Em 1996, o mundo gerou, utilizando tecnologia de restrições um valor estimado de 100 milhões de dólares \cite{fruewirth2003}.


\textcolor{red}{refazer as secoes anteriores}


\section{Complexidade}


\textcolor{red}{Refazer o Conteudo anterior e acrescentar o tema sobre complexidade e problemas P e NP e com isto o que a PR resolve}

\section{Árvores de Buscas: Ilustrando a Complexidade}

\textcolor{red}{acrescentar conteúdo dos slides do curso ... }



\section{Conclusões Parciais}
{\bf \textcolor{red}{Faltando .... e melhorar a apresentacao acima}}
