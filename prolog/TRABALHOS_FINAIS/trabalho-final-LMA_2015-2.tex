\documentclass[a4paper,12pt]{article}
\usepackage[utf8]{inputenc}
\usepackage[brazilian]{babel}
\usepackage[T1]{fontenc}
\usepackage[a4paper,left=30mm,right=30mm,top=15mm,bottom=15mm]{geometry}
\usepackage{graphicx,url}
\usepackage{color,comment, pifont}
\usepackage{amssymb}

\begin{comment}
\usepackage[T1]{fontenc}
\usepackage[utf8]{inputenc}
\usepackage{lmodern}
\usepackage[francais]{babel}
\end{comment}

\title{Lógica Matemática -- Trabalho Final -- 2015-2}
\author{Claudio Cesar de Sá}
\date{\today}

\graphicspath{{/figures/}}   
\DeclareGraphicsExtensions{{.jpg},{.png}}


\begin{document}
\maketitle

\begin{flushleft}





\vspace{0.5cm}
\ding{224}  {\bf \textcolor{red}{Este semestre, o enunciado dos problemas voces vão ler no site oficial, de onde os problemas se propõem.}}


\vspace{0.5cm}
\ding{224} {\bf \textcolor{blue}{Leiam atentamente as
instruções que se seguem.}}


\vspace{0.5cm}
\ding{224} Tarefa: Implementar \textbf{02} (dois) dos  \textbf{03} (três)  problemas dos problemas propostos abaixo.


\vspace{0.5cm}
\ding{224} Entrega pelo site: \textcolor{red}{\url{https://dropitto.me/Trabalhos_Finais_de_LMA_2015_2}}

\vspace{0.5cm}
\ding{224} A senha de entrega deste ano é: lma20152

\vspace{0.5cm}
\ding{224} Entrega dos trabalhos: \textcolor{red}{20/junho} (para o 1o. Semestre)\\
\textcolor{red}{20/novembro} (para o 2o. Semestre). Em geral, pode-se
ocorrer uma flexibilização aqui.\\

\vspace{0.5cm}
\ding{224} Todas estas datas foram escritas no primeiro dia letivo de aula
do semestre corrente junto com datas de provas.


\vspace{0.5cm}
\ding{224} Implementação em SWI-Prolog ou Eclipse (www.eclipseclp.org) (ver a apostila do curso)\\

\vspace{0.5cm}
\ding{224} \textcolor{red}{\textbf{Quanto aos nomes dos arquivos a serem enviados}}:
\begin{itemize}
  \item Não envie os arquivos compactados (serão automaticamente excluídos)
  \item Envie os arquivos  via site: \url{https://dropitto.me/Trabalhos_Finais_de_LMA_2015_2}
  \item Não use email
  \item O nome do arquivo deste deve conter: seu nome,
  sua turma, e o problema resolvido, extensão pode ser txt, pl, ecl etc.
  \item Não coloque espaços em brancos nos nomes do problemas. Use o \_ para ligar nomes
  \item Exemplo de nome de uma arquivo:  \textbf{joao\_silva\_e\_pedro\_souza\_TB\_problema\_as\_estrelas.txt}
  \item Dentro dos códigos coloque o seu nome também.
\end{itemize}


\vspace{0.5cm}
\ding{224} Além dos códigos, sob forma de cometários as entradas e saídas com os testes de seus programas. Estas
entradas e saídas devem vir COMENTADAS no código fonte.

\vspace{0.5cm}
\ding{224} Os testes exaustivos no próprio código fonte vão demonstrar que seu programa está fazendo o que se solicita.

\vspace{0.5cm}
\ding{224} Inclua a saídas do programa e seu tempo de execução (\textbf{\textcolor{green}{isto vai assegurar que não existam cópias de código}}). Há um exemplo de como se calcula tempo de execução, ver código: \textbf{hexagono\_19.ecl}

\vspace{0.5cm}
\ding{224} Alguns fontes e materiais de apoio (incluindo este enunciado) estão em: \textbf{\url{https://github.com/claudiosa/prolog}} 

\vspace{0.5cm}
\ding{224} \textbf{\textcolor{magenta}{Não se impressione pela classificação da dificuldade do problema no site. O que é difícil para o homem, pode ser fácil para máquina!}}

\end{flushleft}

\begin{enumerate}
\setlength\itemsep{0.1cm}
\item Dicas de como se resolve manualmente:\\
\url{http://www.valdiraguilera.net/problema-de-logica-esquema.html}

\item Há exemplos detalhados em  \url{https://github.com/claudiosa/prolog}

\item Use a lista da disciplina para as dúvidas ou procure os professores \textbf{pessoalmente}

\item Para que o \textit{código de honra} (evitar cópias de trabalhos) seja mantido, troquem os nomes dos personagens das estórias abaixo, por seus nomes e/ou de suas família/amigos etc. 

\end{enumerate}
%%%%%%%%%%%%%%%%%%%%%%%%%%%%%%%%%%%%%%%%%%%%%%%%%%%%%%%%%%%%

\newpage

\begin{center}
\fbox{\fbox{{\Huge \textbf{\hskip 1cm AVISO\hskip 1cm}}}}

\vskip 2cm
{\Large
Para todos quando formos ao laboratório: \textbf{\textcolor{magenta}{nem pensem em atacar estes problemas de imediato}}. Poderá ser frustrante para alguns. Voces deverão começar com os exercícios de sala de aula e os do site. \textbf{Um passo de cada vez !}
}
\vskip 2cm
\end{center}

Algumas fontes alternativas de aprendizado s\~ao:

\begin{enumerate}

%\item  \url{https://www.dropbox.com/home/cursos/lma/exercicios_prolog}

\item Alguns outros Prologs:  \url{http://www.thefreecountry.com/compilers/prolog.shtml}

\item Prolog on-line: \url{http://www.tutorialspoint.com/execute_prolog_online.php}. Simplesmente: \textbf{\textcolor{magenta}{Fantástico!}}

 \item No seu telefone (\emph{smartphone}) instale: Jekejeke Prolog (nenhuma semelhança com o time local), tanto faz o Runtime ou o Development (este vem com 
 \emph{debugger}, ótimo para aprender de verdade)
 
 \item Alguns exemplos no meu GitHub de Prolog

\end{enumerate}





\newpage
\tableofcontents


%%%%%%%%%%%%%%%%%%%%%%%%%%%%%%%%%%%%%%%%%%%%%%%%%%%%%%%%%%%%%%%%%%%%%%%%%%%%%
\newpage
\section{Laboratório de Informática}

 Fonte do problema proposto:\\
 \url{http://rachacuca.com.br/logica/problemas/laboratorio-de-informatica/}
 (tem a montagem da tabela para irem entendendo e depurando o problema).\\


\vspace{1.5cm}
\ding{224} Sua tarefa é associar todas essas informações a partir dessas dicas dadas e deduzir o que problema solicita. Acompanhe o andamento de sua solução pela fornecida no site.
%%%%%%%%%%%%%%%%%%%%%%%%%%%%%%%%%%%%%%%%%%%%%%%%%%%%%%%%%%%%%%%%%%%%%%%%%%%%%
\newpage
\section{Turistas na Copa}

 Fonte do problema proposto:\\
 \url{http://rachacuca.com.br/logica/problemas/turistas-na-copa/}
 (tem a montagem da tabela para irem entendendo e depurando o problema).\\


\vspace{1.5cm}
\ding{224} Sua tarefa é associar todas essas informações a partir dessas dicas dadas e deduzir o que problema solicita. Acompanhe o andamento de sua solução pela fornecida no site.
%%%%%%%%%%%%%%%%%%%%%%%%%%%%%%%%%%%%%%%%%%%%%%%%%%%%%%%%%%%%%%%%%%%%%%%%%%%%%
\newpage
\section{Passeio de Bicicleta}

 Fonte do problema proposto:\\
 \url{http://rachacuca.com.br/logica/problemas/passeio-de-bicicleta/}
 (tem a montagem da tabela para irem entendendo e depurando o problema).\\

\vspace{1.5cm}
\ding{224} Sua tarefa é associar todas essas informações a partir dessas dicas dadas e deduzir o que problema solicita. Acompanhe o andamento de sua solução pela fornecida no site.
%%%%%%%%%%%%%%%%%%%%%%%%%%%%%%%%%%%%%%%%%%%%%%%%%%%%%%%%%%%%%%%%%%%%%%%%%%%%%

\end{document}
