\documentclass[10pt,a4paper]{article}
\usepackage{t1enc}
\usepackage[latin1]{inputenc}
\usepackage[portuges]{babel}
\pagestyle{empty}
\markright{}
\topmargin -1.5cm
\textwidth17cm
\textheight26cm
\setlength{\parskip}{0.5cm}
\setlength{\parindent}{1cm}
\oddsidemargin0cm
\begin{document}
\framebox[\textwidth][c]{{\large 1a. Prova - LPG- I - \today}}
\vskip13pt
\noindent Aluno: \hrulefill
\begin{enumerate}
\item Explique e exemplifique passagem por refer�ncia;
\item Qual a direferen�a entre  macros e fun��es, suas vantagens e desvantagens, e exemplifique;
\item Dado um vetor contendo  40 car�cteres, e outros dois  com o comprimento de 20 caracteres. Fa�a uma fun��o que passe esses tres vetores, tal que ao final, os dois vetores menores, tenham partes do vetor maior. A divis�o do vetor maior pode ser intercalada ou pela metade;

\item Dada a estrutura abaixo:
\begin{verbatim}
   struct  reg
               {
                char nome[30];
                char sexo;
                int idade;
               };
e
             struct reg     vetor[10];
\end{verbatim}
Considere que  a vari�vel  ``{\em vetor}"  contenha alguns registros preenchidos. Registros vazios tem 0 no campo idade com identificador. Fa�a as seguintes fun��es:
\begin{enumerate}
\item Por passagem por refer�ncia, um procedimento que informe quantos homens e mulheres existem no vetor;
\item Uma fun��o de devolva o peso m�dio de todos registros v�lidos;
\end{enumerate}

\item Definindo um ponteiro � estrutura acima, fa�a uma fun��o que receba um ponteiro e um nome a ser pesquisado. Caso nome esteja no  vetor, retorne `S', caso contr�rio 'N';

\item Dado um vetor homog�neo do tipo ``{\em float}", crie e inicialize  um ponteiro a este vetor. Fa�a uma fun��o, passando o ponteiro em quest�o, que verifique o conte�do de cada c�lula do vetor, fazendo os seguintes c�lculos:
\begin{enumerate}
\item Se o valor for ``$>= 10$", ent�o multiplique por 0.5;
\item Caso contr�rio, divida por 0.5. 
\end{enumerate}

\item Utilizando-se de um vetor de ponteiros, inicializado da seguinte maneira:
\begin{verbatim}
   static   char *X[] = {          "Incluir",
                                  "Alterar",
                                  "Excluir",
                                  "Sair"};
\end{verbatim}
Organize um la�o repetitivo com  ``{\em switch}",  tal que imprime este menu de op��es (obviamente que via ponteiros), e um teste de consist�ncia para verificar se um dos car�cteres \{ 'I', 'A', 'E', 'S'\} foi digitado. As compara��es s�o via ponteiros.

\item Dado um programa que recebe via linha de comando uma sequ�ncia de n�meros quaisquer. Fa�a esse programa e que ao final, seja encontrado o maior n�mero digitado.
\end{enumerate}
\vskip 13pt
PS. Clareza e legibilidade!
\end{document}