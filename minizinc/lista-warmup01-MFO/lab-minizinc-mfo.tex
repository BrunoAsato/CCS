\documentclass[paper=a4, fontsize=11pt]{scrartcl} % A4 paper and 11pt font size

\usepackage[T1]{fontenc} % Use 8-bit encoding that has 256 glyphs
\usepackage{fourier} % Use the Adobe Utopia font for the document - comment this line to return to the LaTeX default

\usepackage[brazilian]{babel}
\usepackage[utf8]{inputenc}

\usepackage{amsmath,amsfonts,amsthm,url} % Math packages

%\usepackage{lipsum} % Used for inserting dummy 'Lorem ipsum' text into the template

\usepackage{sectsty} % Allows customizing section commands
\allsectionsfont{\centering \normalfont\scshape} % Make all sections centered, the default font and small caps

\usepackage{fancyhdr} % Custom headers and footers
\pagestyle{fancyplain} % Makes all pages in the document conform to the custom headers and footers
\fancyhead{} % No page header - if you want one, create it in the same way as the footers below
\fancyfoot[L]{Exercícios de MiniZinc -- 2015} % Empty left footer
\fancyfoot[C]{} % Empty center footer
\fancyfoot[R]{\thepage} % Page numbering for right footer
\renewcommand{\headrulewidth}{2pt} % Remove header underlines
\renewcommand{\footrulewidth}{2pt} % Remove footer underlines
\setlength{\headheight}{1cm} % Customize the height of the header

\numberwithin{equation}{section} % Number equations within sections (i.e. 1.1, 1.2, 2.1, 2.2 instead of 1, 2, 3, 4)
\numberwithin{figure}{section} % Number figures within sections (i.e. 1.1, 1.2, 2.1, 2.2 instead of 1, 2, 3, 4)
\numberwithin{table}{section} % Number tables within sections (i.e. 1.1, 1.2, 2.1, 2.2 instead of 1, 2, 3, 4)

\setlength\parindent{0pt} % Removes all indentation from paragraphs - comment this line for an assignment with lots of text

%----------------------------------------------------------------------------------------
%	TITLE SECTION
%----------------------------------------------------------------------------------------

\newcommand{\horrule}[1]{\rule{\linewidth}{#1}} % Create horizontal rule command with 1 argument of height

\title{	
\normalfont \normalsize 
\textsc{{\huge UDESC -- CCT -- DCC}} \\ [25pt] % Your university, school and/or department name(s)
\horrule{2pt} \\[0.4cm] % Thin top horizontal rule
\huge Laboratório de Minizinc em MFO \\ % The assignment title
\horrule{2pt} \\[0.5cm] % Thick bottom horizontal rule
}

\author{Lucas e Claudio} % Your name

\date{\normalsize\today} % Today's date or a custom date

\begin{document}

\maketitle % Print the title


\section{Objetivo da Lista}

Utilizar a linguagem de modelagem MiniZinc em resoluções
de problemas da teoria dos conjuntos, funções, relações, 
lógica proposicional e lógica  primeira-ordem.\\

Fonte de referência: \url{https://github.com/claudiosa/minizinc}

\tableofcontents

%----------------------------------------------------------------------------------------

%----------------------------------------------------------------------------------------
\newpage
\section{Operações sobre Conjuntos}
%https://oeis.org/wiki/List_of_LaTeX_mathematical_symbols
%------------------------------------------------

\subsection{União}

Construa um código que realize a união dos conjuntos abaixo:
\begin{enumerate}

\item $A = \{ 1, 2, 4, 6\}$ e $B = \{ 4, 3, 7, 8, 9 \}$

\item $A = \{ 0, -1, 1, 5\}$ e $B = \{ 0, -5, 10, 8, 3 \}$


\end{enumerate}


\subsection{Interseção}

Construa um código que realize a interseção dos conjuntos abaixo:
\begin{enumerate}

\item $A = \{ 1, 2, 4, 6\}$ e $B = \{ 4, 3, 7, 8, 9 \}$

\item $A = \{ 0, -1, 1, 5\}$ e $B = \{ 0, -5, 10, 8, 3 \}$


\end{enumerate}



\subsection{Diferenças}

Construa um código que realize a diferença dos conjuntos abaixo:
\begin{enumerate}

\item $A = \{ 1, 2, 4, 6\}$ e $B = \{ 4, 3, 7, 8, 9 \}$

\item $A = \{ -5, -3, 2, 5\}$ e $B = \{ 0, -5, 10, 8, 3 \}$


\end{enumerate}




%------------------------------------------------------------------------------------------------
\newpage
\section{Relações}

\subsection{Funções}

Construir funções que calculem:

\begin{enumerate}

\item  A sequência de fibonacci, até um número N (pode ser especificado no código).

\item Os N primeiros números primos.


\end{enumerate}


\subsection{Tuplas}

\begin{enumerate}

\item  Calcule o Conjunto C, onde cada elemento de C é uma dupla (x,y), x é o primeiro elemento da n-ésima dupla de A e y é o segundo elemento da n-ésima dupla de B.  $A = \{(0,5),(1,7),(-3,5),(-5,8)\}$ e $B = \{(-1,0),(0,-1),(5,4),(,2,9),(-7,-6) \}$.

\item O produto cartesiano inverso dos conjuntos: $A = \{ -5, -3, 2, 5\}$ e $B = \{ 0, -5, 10, 8, 3 \}$.


\end{enumerate}




%----------------------------------------------------------------------------------------



%------------------------------------------------
\newpage
\section{Lógica Proposicional -- LP}
%----------------------------------------------------------------------------------------

Comprove os teoremas lógicas abaixo:

\begin{enumerate}

%\item $(p \land (p \to q))\vdash q$ 

\item $((p \to q )\land (p \to r))\vdash p \to r$

%\item $(p \to q )\vdash( p \to ( p \land q))$

\item $(p \lor q)\vdash( p \to ( p \land q))$

\item $((p \to q) \land (r \to g)) \vdash((p \lor r) \to ( q \lor g))$

\item $((p \to q) \land (r \to g)) \vdash((\lnot q \lor \lnot g) \to(\lnot p \lor \lnot r))$

\end{enumerate}


%------------------------------------------------
\newpage
\section{Lógica Primeira Ordem -- LPO}
%----------------------------------------------------------------------------------------

Seja o exemplo de uma fórmula de LPO:\\
$$\exists x . \forall y .(x \ge y)$$
Leia-se: \textit{existe um $x$ para  todos $y$ tal que $x \ge y$}\\
seu código equivalente é dado por:
\begin{small}
\begin{verbatim}
constraint %%% Existe um x para  todos y tal que x>=y 
 exists(i in 1..n)(
  forall  (j  in 1..m )( 
   (x[i] >= y[j]) <-> (Phi03 == true) %% Apenas interpretacoes TRUE
%%% OU  (y[j] > x[i]) <-> (Phi03 == true)
   ) 
   );
\end{verbatim}
\end{small}

Cujos domínios são:
\begin{verbatim}
x = [1, 3, 5];
y = [2, 4, 6, 8];
array[1..n] of int : x;
array[1..m] of int : y;
\end{verbatim}

\subsection{Interpretação}

Encontre as interpretações verdadeiras para as fórmulas abaixo:
%%%∃x.(Computer(x) ∧ ∀y.(Student(y) → ¬Uses(y, x)))
\begin{enumerate}

\item $\forall x . \exists y .(2x - y = 0)$
\begin{verbatim}
x = [0, 2, 4];
y = [5, 3, 0, 8, 4, 2, 6];
array[1..n] of int : x;
array[1..m] of int : y;
\end{verbatim}

\item $\exists x .(computador(x) \wedge \forall y.(estudante(y) \rightarrow \sim usa(y, x)))$
\begin{verbatim}
x = [1, 2, 3, 4, 5, 6];
y = [1, 2, 3, 4, 5, 6];
\end{verbatim}

\item Nenhum estudante(x) reprovou em y, mas ao menos 3 estudantes reprovaram em z. Em LPO temosa seguinte fórmula:\\

$\sim\:\: \exists x \forall y . (reprovou(x,y)) \land \exists x \forall y .(reprovou(x,z) \land x \ge 3)$

\begin{verbatim}
x = [1, 2, 3, 4, 5, 6];
y = [2, 3]; 
z = [3, 4]; 
\end{verbatim}

\end{enumerate}
\end{document}
