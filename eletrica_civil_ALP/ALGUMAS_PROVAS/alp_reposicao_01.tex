\documentclass[12pt]{article}
\usepackage[a4paper,left=25mm,right=25mm,top=15mm,bottom=25mm]{geometry}
\usepackage{graphicx,url}
\usepackage{color}
\usepackage{amssymb,listings}
\usepackage[utf8]{inputenc}


% Setting configuration for the text format
%\renewcommand{\contentsname}{Table of Contents}
%\renewcommand{\bibname}{References}
%\titleformat{\chapter}[display]{\normalfont\huge\bfseries}{\filleft\chaptername\ \thechapter}{5pt}{\filleft\Huge}
%\sloppy

\title{Algoritmos e Linguagens de Programação --  Reposição 01 -- Ajuda na Nota}
\author{ Claudio Cesar de Sá}
\date{\today}

\graphicspath{{/img/}}   
\DeclareGraphicsExtensions{{.jpg},{.png}}

\lstset {
  numbers=left,
}

\begin{document}
%\pagestyle{empty}
\maketitle



\begin{flushleft}
\textbf{Nome}: \rule{10cm}{0.3mm} 
\end{flushleft}
%\textbf{Matrícula}: \rule{7cm}{0.3mm} 
% %\rule{\linewidth}{1.
\begin{enumerate}
%\setlength{\itemsep}{-1pt}

\item Considere a nota de N  ($1 \le N \le 100$)  alunos as quais são dadas 03
por três números reais. Para cada um dos casos 
calcule a média  de TODAS entradas, e indique na saída um dos seguintes casos:

\begin{enumerate}
  \item media $< 2.0$ : Volte semestre que vem!
    \item $2.0 \le$ media $< 7.0$ : Em exame!
        \item media $\ge 7.0$ : Se safou!
\end{enumerate}
Como  casos de testes, considere o seguinte arquivo de entrada:
\begin{verbatim}
10
3.5 7.6 8.5
4.5 8.56 8.9
6.7 8.1 9.1
5.90 7.6 8.56
4.5 8.56 8.19
7.2 8.1 9.1
9.1 7.1 6.1
\end{verbatim}
Quantos alunos passaram por média?


\item Usando a estrutura \texttt{switch-case}, combinada com \texttt{do-while}, construa um programa em C que mostre o seguinte menu na tela:
\begin{small}
\begin{verbatim}
Cadastro de Clientes
0 - Fim
1 - Inclui
2 - Altera
3 - Exclui
4 - Consulta
Opção: 
\end{verbatim}
\end{small}

Para cada uma das opções acima de 1 a 4 escreva a mensagens da opção. Se digitar 0,
encerra o programa, e qualquer outro número, imprima a mensagem: ''\texttt{Digite alguma opção
válida}"\/ e encerre o programa.

\item  Escreva um programa em C que leia N ($1 \le N \le 100$)  casos, onde cada caso é composto por
 3 números inteiros: $x$, $y$ e $m$, em seguida identifique se os dois números  $x$ e $y$ são congruentes
entre si dado o \textit{módulo} $m$. Por definição
a congruência é definida por:

$$ x \% m = k$$
$$ y \% m = k$$
 
 Exemplo: 35 \% 4 = 3  e   39 \% 4 = 3. Módulo é o resto da divisão inteira, na linguagem 
 simbolizada por \%. Neste exemplo, o números 35, 39 são congruentes entre si
 pelo \textit{módulo} 4. Imprima as saídas: \texttt{congruentes} ou \texttt{incongruentes} conforme o caso.

Como  casos de testes, considere o seguinte arquivo de entrada:
\begin{verbatim}
10
5 7 8
54 76 87
4 76 8
54 76 871
5 61 7
14 76 8
15 6 7
\end{verbatim}




\item Escreva um programa em C que determine quanto tempo transcorreu entre duas marcações de tempo. São dados de entrada: $tempo_1 = HH:MM:SS$ e $tempo_2 = hh:mm:ss$. 
\\Exemplo: entre as marcações $tempo_1 = 10:05:47$ e $tempo_2 = 21:00:01$ transcorreram $10:54:14$

{\footnotesize DICA: O operador divisão (/) quando aplicado a números inteiros, retorna o quociente da divisão.}
x
\end{enumerate}

\end{document}
