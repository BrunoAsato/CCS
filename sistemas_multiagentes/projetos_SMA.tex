


\section{Projetos de SMAs}

\begin{frame}

\begin{center}
{\huge Capítulo 5 -- Projetos de SMAS}
\end{center}

\end{frame}


%------------------------------------------------------------------------
\begin{frame} %[allowframebreaks=0.9]


\frametitle{O que vai ter neste capítulo}

\begin{itemize}
  \item Alguma metodologia?
  \item Ambiente simulado: SIM (implementar)
  
  \item O que considerar na construção de SMAs
  \item Perspectivas: reais e visionárias
\end{itemize}


\end{frame}

%-----------------------------------------------------------------------------------

\section{Implementação de SMAs}
\begin{frame}

    \frametitle{Agentes: Metodologia de desenvolvimento}
    \begin{itemize}
    \pause
      \item Decompõem o problema em: \\
      percepções, ações, objetivos, ambiente e outros agentes
\pause
      \item Decompõem tipo de conhecimento em:
      \begin{itemize}
        \item Quais são as propriedades relevantes do mundo?
        \item Como o mundo evolui?
        \item Como identificar os estados desejáveis do mundo?
        \item Como interpretar suas percepções?
        \item Quais as conseqüências de suas ações no mundo?
        \item Como medir o sucesso de suas ações?
        \item Como avaliar seus próprios conhecimentos?
    
      \end{itemize}
      
      \item O resultado dessa decomposição indica a arquitetura e o método de resolução de problema (raciocínio)
      
    \end{itemize}
\end{frame}

%-----------------------------------------------------------------------------------

\subsection{Simulação de Ambientes}
\begin{frame}

    \frametitle{Simulação de Ambientes (todos elementos de um SMA)}
    \begin{itemize}
    \pause
      \item Muitas vezes, é mais conveniente simular o ambiente (problema): \\
     \begin{itemize}
        \item  mais simples
        \item  permite testes prévios
        \item  evita riscos, etc...
                
        \item  genuinamente: prototipação
      \end{itemize}
\pause
      \item O ambiente (programa):
      \begin{itemize}
        \item recebe os agentes como entrada
        \item fornece repetidamente a cada um deles as percepções corretas e recebe as ações
        \item atualiza os dados do ambiente em função dessas ações e de outros processos (ex. dia-noite)
        \item é definido por um estado inicial e uma função de atualização
        \item deve refletir a realidade
    
      \end{itemize}
      
      
    \end{itemize}
\end{frame}
%-----------------------------------------------------------------------------------




%-----------------------------------------------------------------------------------

\begin{frame}

    \frametitle{Simulação de Ambientes}

     ESTRUTURAS ... ver NORVIG -- pg 34 da T

\end{frame}
%-----------------------------------------------------------------------------------








\begin{comment}
\sectio{Aprendizagem}
\begin{frame}

    \frametitle{Aprendizagem}
    \begin{itemize}
    \pause
      \item 
\pause
      \item cap 7
    
    \end{itemize}
\end{frame}
\end{comment}
